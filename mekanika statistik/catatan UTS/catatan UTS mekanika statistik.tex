\documentclass[a4paper  , 6 pt]{article}
\usepackage[fleqn]{amsmath}
\usepackage{hyphenat}
\usepackage{setspace}
\usepackage{anysize}
\usepackage{parskip}
\usepackage{multicol}
\usepackage{blindtext}
\usepackage{mathrsfs}
\usepackage{amssymb}
\usepackage{mathtools}
%\usepackage{enumitem}
\usepackage{tikz}

%\marginsize{0.3 cm}{0.3 cm}{0 cm }{0 cm }
\setlength{\mathindent}{0 cm }
%\setlength{\}{length}
\usepackage[margin={0.2 cm ,0.2 cm}]{geometry}
%\setitemize[0]{leftmargin=*}
%\usepackage[margin =2.5 cm]{geometry }
%\title{PR 2 MEKANIKA STATISTIK}
%\author{MOHAMMAD FAJAR \newline 20211019}
\singlespacing
\pagestyle{empty}
\allowdisplaybreaks
%\pagestyle{empty}

\begin{document}

%\begin{multicols*}{2}
\begin{tiny}
\begin{multicols} {5}

\setlength \belowdisplayskip{0 pt} 
\setlength \abovedisplayskip{0 pt}
%\underline{\textbf{Statistik Klasik  terbedakan }}
\textbf{Persamaan-persamaan yang sering digunakan } \newline Bobot konfigurasi statistik fermi dirac dinyatakan oleh:
\begin{align}
W = \prod_s w_s 
\end{align}
dengan 
\begin{align}
w_s = \frac{g_s !}{n_s ! (g_s - n_s) !} 
\end{align}
Untuk bose-einstein dinyatakan oleh
\begin{align}
W& = \prod_s w_s \nonumber \\
& = \prod_s \frac{[(g_s - 1) + n_s] !}{(g_s - 1)! n_s ! }
\end{align} 
Untuk Boltzmann:
\begin{align}
W = N! \prod_s \left \lbrace \frac{g_s^{n_s}}{n_s !}\right \rbrace 
\end{align}
Untuk statistik semi-klasik 
\begin{align}
W =  \prod_s \left \lbrace \frac{g_s^{n_s}}{n_s !}\right \rbrace 
\end{align}
Fungsi distribusi fermi-dirac dinyatakan oleh
\begin{align}
\overline{n}_\mathrm{FD} =  \frac{1}{e^{(\epsilon - \mu )/kT} +1} 
\end{align}
Fungsi distribusi bose-einstein dinyatakan oleh 
\begin{align}
\overline{n}_\mathrm{BE} =  \frac{1}{e^{(\epsilon - \mu )/kT} - 1} 
\end{align}
Fungsi distribusi Boltzmann dinyatakan oleh
\begin{align}
\overline{n}_\mathrm{boltzmann} &= e^{\mu/ kT} e^{- \epsilon / kT} \nonumber \\
& = e^{- (\epsilon - \mu )/ kT} 
\end{align}           
Energi bebas Helmoltz dinyatakan oleh
\begin{align}
F = - kT \, \ln Z 
\end{align}
dengan demikian diperoleh
\begin{align}
& S = -\left(\frac{\partial F}{\partial T}\right)_{V,N}, \\
& P  = - \left(\frac{\partial F}{\partial V}\right)_{T,N}, \\
 & \mu = + \left(\frac{\partial F}{\partial N}\right)_{T,V}
\end{align}
\begin{align}
S = k \ln W
\end{align}
\begin{align}
\frac{\mathcal{F}(s_2) }{\mathcal{F} (s_1)} =  \frac{\Omega (s_2)}{\Omega (s_1)}
\end{align}
\begin{align}
\mathcal{P} (s) = \frac{1}{Z}e^{-E_s/kT} 
\end{align}
\begin{align}
\frac{N_s}{N} = \frac{e^{-E(s)/kT}}{Z}
\end{align}
\begin{align}
\overline{E} & = \frac{\sum_s E(s) N(s)}{N} = \sum_s E(s) \frac{N(s)}{N} \nonumber \\
& = \sum_s E(s) \mathcal{s} = \frac{1}{Z} \sum_s E(s) e^{-\beta E(s) }
\end{align}
\textbf{Degenerasi gas fermi.} \newline Jumlah Keadaan energi dalam rentang energi  antara $\epsilon$ hingga $\epsilon+ d \epsilon$ adalah
%\setlength \abovedisplayskip{0 pt}
\begin{flalign}
g(\epsilon)d\epsilon = \frac{2 \pi (2m )^{3/2} \epsilon^{1/2} d \epsilon \cdot V}{h^3}
\end{flalign}
karena fermion terdapat dua keadaan yang mungkin $-1/2$ dan $1/2$ maka nilai ini dikalikan 2 sehingga diperoleh:
%\setlength \abovedisplayskip {0 pt}
\begin{align}
g(\epsilon) = V \cdot 4 \pi \left(\frac{2m}{h^2}\right)^{3/2} \epsilon^{1/2} \label{density of states}
\end{align}
Untuk $\epsilon < \epsilon_\mathrm{F}(0)$ maka 
\begin{align}
f(\epsilon) = \frac{1}{e^{- \infty} + 1} = 1
\end{align}
untuk $\epsilon > \epsilon_\mathrm{F}(0)$ maka 
\begin{align}
f(\epsilon) = \frac{1}{e^{\infty} + 1} = 0
\end{align}
Sehingga untuk memperoleh nilai energi fermi, maka persamaan ini Mesti diintegralkan dari 0 sampai $\epsilon_F$ 
%\setlength \abovedisplayskip{0 pt}
\begin{align}
\int_0^{\epsilon_\mathrm{F}} V \cdot 4 \pi \left(\frac{2m}{h^2}\right)^{3/2} \epsilon^{1/2} d \epsilon = N 
\end{align}
yang hasilnya adalah 
%\setlength \abovedisplayskip{0 pt}
\begin{align}
\epsilon_\mathrm{F} (0)= \frac{h^2}{2m }\left(\frac{3N}{8 \pi V}\right)^{3/2}
\end{align}
Temperatur fermi selanjutnya dapat diperoleh melalui hubungan:
\begin{align}
kT_\mathrm{F}(0) =  \epsilon_\mathrm{F} (0)
\end{align}
kemudian energi rata-rata elektron dapat dinyatakan sebagai:
\begin{align}
\overline{\epsilon} (0) = \frac{\int_{0}^{{\epsilon_\mathrm{F}(0)}}\epsilon g(\epsilon) d \epsilon}{\int_{0}^{{\epsilon_\mathrm{F}(0)}} g(\epsilon) d \epsilon} = \frac{3}{5} \epsilon_\mathrm{F} (0) 
\end{align}
%\end{multicols}
Dengan cara yang sama dapat kita hitung momentum rata-rata dan kecepatan rata-rata yakni dengan melakukan transformasi $\epsilon = p^2/2m$. 
Energi total dinyatakan oleh
\begin{align}
U = N \overline{\epsilon } = \frac{3}{5} N \epsilon_\mathrm{F}
\end{align}
Sehingga persamaan keadaan dapat diperoleh yakni
\begin{align}
P & = - \frac{\partial }{\partial V} \left[ \frac{3}{5} N \frac{h^2}{2m} \left( \frac{3N}{8 \pi }\right)^{3/2} \right. \nonumber \\
& \hspace{0.2 cm } \left. \cdot V^{-3/2}  \right ] \nonumber \\
& = \frac{2 N \epsilon_\mathrm{F}}{5 V} = \frac{2 U}{3 V} \label{pers.keadaan fermi dirac}
\end{align} 
%Untuk temperatur di atas nol mutlak maka
\textbf{Statistik bose-einstein: Tinjauan radiasi benda hitam} 
\newline
Jumlah mode gelombang yang dibolehkan untuk panjang gelombang $\lambda$ sampai $d \lambda$ dinyatakan oleh
\begin{align}
g(\lambda) d \lambda = \frac{4 \pi}{\lambda^4} d \lambda \label{persamaan gelombang em}
\end{align}
Karena dalam gelombang elektromagnetik terdapat dua polarisasi yang saling tegak lurus maka nilai ini mesti dikalikan dengan 2 sehingga diperoleh:
\begin{align}
g(\lambda) d \lambda = \frac{8 \pi}{\lambda^4} d \lambda 
\end{align}
perhatikan: persamaan (6) setara dengan persmaan (1)
%\end{multicols} 
\newline
sementara distribusi bose-einstein dinyatakan oleh:
\begin{align}
n_s = \frac{g_s}{e^{h\nu_s /kT} -1}
\end{align}
Sehingga jumlah foton, $n_\lambda (\lambda) d\lambda $ untuk rentang panjang gelombang antara $\lambda $ hingga $\lambda  + d \lambda$ dinyatakan oleh:
\begin{align}
n_\lambda (\lambda) d \lambda = \frac{8 \pi}{\lambda^4} d \lambda \cdot \frac{1}{e^{hc/ k\lambda T} - 1}
\end{align}
Di mana telah disubstitusi $h \nu  = h c/ \lambda$.
Energi $E(\lambda) =n_\lambda (\lambda) h \nu = n_\lambda (\lambda) h c/ \lambda  $ dapat dinyatakan oleh:
\begin{align}
E(\lambda) d \lambda = \frac{8 \pi h c \, d\lambda}{\lambda^5 (e^{hc/k \lambda T} - 1)} 
\end{align}
Untuk panjang gelombang yang besar, maka $e^{hc /k \lambda T} \simeq 1+ hc/ k\lambda T$ sehingga diperoleh
\begin{align}
E(\lambda ) d\lambda  \simeq \frac{8 \pi k T \, d \lambda}{\lambda^4}
\end{align}
yang sama dengan formulasi Rayleigh Jeans. 
Sementara untuk panjang gelombang yang rendah di mana $e^{hc/k \lambda T} \gg 1$ akan diperoleh:
\begin{align}
E(\lambda ) d\lambda  \simeq \frac{8 \pi h c}{\lambda^5} e^{- hc/k \lambda T} \, d \lambda 
\end{align}
yang merupakan formulasi distribusi Wein.
%\end{multicols}
 Dari persamaan (11) dapat diperoleh total energi persatuan volume yang terlingkupi dalam benda hitam tersebut yakni:
 \begin{align}
 E& = \int_{0}^{\infty} E(\lambda) d \lambda \nonumber \\
 & = \int_{0}^{\infty} \frac{8 \pi h c d \lambda}{\lambda^5 (e^{hc /k \lambda T} -1)} \nonumber \\
 & = \frac{8 \pi h}{c^3} \left \lbrace \frac{kT}{h} \right \rbrace^4 \int_{0}^{\infty} \frac{t^3 \, dt }{e^t - 1}  \label{energi}
 \end{align}
 Dengan 
 \begin{align}
 \int_{0}^{\infty} \frac{t^3}{e^t -1} = 6 \sum_{n = 1}^\infty \frac{1}{n^4} = \frac{\pi^4}{15} 
 \end{align}
 Maka persamaan \ref{energi} dapat dituliskan menjadi 
 \begin{align}
\boxed{E = \left \lbrace  \frac{8 \pi^5 k^4 }{15 h^3 c^3 }\right \rbrace  T^4}
 \end{align}
 \newline
\textbf{Gas fonon.} \newline
Jumlah fonon $n(\nu) d\nu$ dengan frekuensi antara $\nu$ sampai $\nu + d \nu$ adalah
\begin{align}
n_\nu(\nu) d\nu = \frac{g(\nu) d\nu}{e^{h\nu /kT} - 1} \tag{13}
\end{align}
Aproksimasi Debye dinyatakan oleh
\begin{align}
g(\nu) d\nu &  = C \nu^2 d \nu , & \nu \le \nu_\mathrm{m} \nonumber \\
g(\nu) d\nu & = 0, & \nu > \nu_\mathrm{m} \nonumber 
\end{align}
Untuk mendapatkan konstanta $C$ maka dilakukan  integrasi terhadap seluruh mode yang mungkin yang nantinya akan menghasilkan nilai $3N$ yakni
\begin{align}
3N & = \int_{0}^{\infty} g(\nu)  d\nu  = \int_{0}^{\nu_\mathrm{m}} C \nu^2  d\nu  \nonumber \\
& = \frac{1}{3} C \nu_\mathrm{m}^3 \nonumber \\
\text{atau }& \nonumber \\
C_\mathrm{m}& = \frac{9N}{v_\mathrm{m}^3} \nonumber
 \end{align}
 sehingga aproksimasi debye memberikan
 \begin{align}
 n_\nu (\nu) d\nu& = \frac{9N}{\nu_\mathrm{m}^3} \frac{\nu^2 d\nu}{e^{h\nu /kT} -  1} ,  \nu \le \nu_\mathrm{m} \nonumber \\
  & = 0,\hspace{0.3 cm}  \nu > \nu_\mathrm{m} \nonumber
 \end{align}
 kemudian 
 \begin{align}
 E & = \int_{0}^{\nu_\mathrm{m}}  h \nu n_\nu (\nu) d\nu  \nonumber \\
 & = \frac{9 N_A h}{\nu_\mathrm{m}^3} \int_{0}^{\nu_\mathrm{m}} \frac{\nu^3 d\nu}{e^{h\nu /kT} - 1} \nonumber
 \end{align}
 kemudian panas spesifik dapat diperoleh yakni
  \begin{align}
  C_\nu & = \left\lbrace \frac{\partial E}{\partial T} \right\rbrace   = \frac{9 N_A h^2}{v_\mathrm{m}^3} \frac{1}{kT^2} \cdot \nonumber \\ 
   & \hspace{0.2 cm }\int_{0}^{\nu_\mathrm{m}} \frac{\nu^4e^{h\nu /kT} d\nu }{(e^{h\nu /kT} - 1)^2} 
 \end{align}
 Jika diadakan perubahan variabel yakni $x = h\nu /kT$ dan $h\nu_\mathrm{m}/k $ diganti oleh $\theta_\mathrm{D}$ maka temperatur karakteristik pada Persamaan di atas dapat dinyatakan menjadi:
 \begin{align}
 C_\nu = 9 R \left \lbrace \frac{T}{\theta_\mathrm{D}} \right \rbrace^3 \int_{0}^{\theta_\mathrm{D}/T} \frac{x^4 e^x d x}{(e^x - 1)^2} \nonumber  
 \end{align}
 Untuk temperatur tinggi, di mana $\theta_\mathrm{D} \ll 1$ maka $e^x \simeq 1+ x \simeq 1 $ sehingga 
 \begin{align}
 C_\nu  \simeq 9 R \left\{ \frac{T}{T_\mathrm{D}} \right \}^3 \int_{0}^{\theta_\mathrm{D}/T} x^2 dx =  3 R \nonumber
 \end{align}
 Untuk temperatur rendah, di mana $e^{- \theta_\mathrm{D} /T}\ll 1$ maka batas atas integrasi diganti menjadi $\infty$ sehingga 
 \begin{align}
 C_\nu
 & \simeq 9 R
 \left\lbrace \frac{T}{\theta_\mathrm{D}}\right\rbrace^3
  \int_{0}^{\infty}
   \frac{x^4 e^x dx}{(e^x - 1)^2} \nonumber 
% \text{ jadi } & \nonumber \\
% C\nu &  \simeq \frac{12}{5} \pi^4  R \left \lbrace  \frac{T}{\theta_\mathrm{D}}\right \rbrace^3 \nonumber 
\end{align}
atau dengan melakukan ekspansi taylor terhadap $ e^x$ maka 
\begin{align}
\int_{0}^{\infty } \frac{x^4 e^x dx }{(e^x - 1)^2 } = 24 \sum_{n =1}^{\infty} \frac{1}{n^4}  = \frac{4 \pi^4}{15} \nonumber
\end{align}
sehingga
\begin{align}
 C\nu &  \simeq \frac{12}{5} \pi^4  R \left \lbrace  \frac{T}{\theta_\mathrm{D}}\right \rbrace^3 \nonumber 
\end{align} \newline
\textbf{Statistik Maxwell Boltzmann} \newline
Untuk sistem partikel terbedakan bobot konfigurasi dinyatakan sebagai
\begin{align} 
W = N! \prod_s \left \lbrace \frac{g_s^{n_s }}{n_s !} \right \rbrace 
\end{align}
distribusi momentum, kecepatan, dan energi dinyatakan oleh
\begin{align}
n_ p dp& =  \frac{4 \pi N }{(2 \pi mk T)^{3/2} } \nonumber \\ & \hspace{0.3 cm} \cdot e^{- p^2 /2 mk T} p^2 dp  
\end{align}
\begin{flalign}
n_v dv& =  4 \pi N \left \lbrace  \frac{m }{2 \pi k T} \right \rbrace^{3/2} \nonumber \\ &\hspace{0.5 cm} \cdot e^{- v^2 /2 mk T} v^2 dv  
\end{flalign}
\begin{align}
	n(\epsilon ) d \epsilon = \frac{2 \pi N }{(\pi kT )^{3/2} } e^{- \epsilon /kT } \epsilon^{1/2} \, d \epsilon 
\end{align}
Dengan demikian, untuk sebuah molekul probabilitas kecepatannya berada dalam selang $v$ hingga $v + dv$  dinyatakan oleh  
\begin{align}
f_v (v) dv& = \frac{n_v (v) dv }{N} \nonumber \\
& = 4 \pi \left \lbrace \frac{m}{2 \pi k T} \right \rbrace^{3/2} \cdot \nonumber \\
&  \exp \left( - mv^2 /2kT\right) v^2 \, dv  \nonumber  
\end{align}
Dari sini selanjutnya dapat diperoleh
\begin{align}
\overline{v}& = \int_{0}^{\infty} v f_v(v) dv  \nonumber \\
& = 4 \pi  \left \lbrace \frac{m}{2 \pi kT} \right \rbrace^{3/2} \int_{0}^{\infty} \cdot \nonumber \\
& \hspace{0.2 cm } \exp \left(- mv^2 /2k T \right) v^3 \nonumber \\
&  = \boxed{\sqrt{\left(\frac{8kT}{\pi m }\right)}}
\end{align}
sementara 
\begin{align}
\overline{v^2}& =  \int_{0}^{\infty} v^2 f_v(v) \, dv \nonumber \\
&  = 4\pi \left \lbrace \frac{m}{2 \pi k T}\right
 \rbrace^{3/2} \nonumber \\
 & \hspace{0.2 cm } \cdot \left(
 \int_{0}^{\infty} \exp\left(-mv^2 /2 kT \right) \right. \nonumber \\
& \left. \hspace{ 0.4 cm }\cdot  v^4 dv \right )\nonumber \\
  &  = \boxed{\frac{3kT}{m} }
\end{align} 
Nilai most probable velocity dinyatakan oleh
\begin{align}
\frac{d f_v (v)}{d v} & = 4 \pi \left \lbrace  \frac{m}{2 \pi k T}\right \rbrace^{3/2} \left \lbrace 2 v - \right. \nonumber \\
& \left. \hspace{0.2 cm } \frac{m v^3}{kT}\right \rbrace  \exp\left( - m v^2 / 2 kT\right) \nonumber \\
& = 0 
\end{align}
yang menghasilkan 
\begin{align}
2  v_m  - \frac{m v_m^3}{kT}  = 0 
\end{align}
atau 
\begin{align}
\boxed{v_m  = \sqrt{\left( \frac{2 kT}{m}\right)}}
\end{align}
\newline
\textbf{Prinsip ekuipartisi energi.} \newline
Energi kinetik dalam satu derajat kebebasan (misal x) dinyatakan oleh $\epsilon_x = p_x^2 /2m $ sehingga 
\begin{align}
\overline{\epsilon_x} = \frac{\int_{\Gamma} p_x^2 /2m \, e^{-\epsilon/kT d \Gamma}}{\int_\Gamma e^{-\epsilon /kT 
} d\Gamma}
\end{align}
Jika energi dinyatakan dalam dua bagian yakni $p_x^2 /2m $ dan $(\epsilon - p_x^2 / 2m)$ makan persamaan di atas dapat dituliskan menjadi
\begin{align}
\overline{ \epsilon_x} & = \left( \frac{\int \exp \left(- \left( \epsilon - \frac{p_x^2}{2m }\right)/kT \right)}{\int \exp \left( - \left( \epsilon - \frac{p_x^2}{2m } \right)/kT\right) } \cdot \right.  \nonumber \\
& \hspace{0.3 cm }  \frac{ dx \, dy \, dz \, dp_y \, dp_z}{dx \, dy \, dz \, dp_y \, dp_z } \cdot \nonumber \\ 
& \hspace{0 cm}
\left.
\frac{\int_{-\infty}^{\infty} \frac{p_x^2}{2m} \exp \left(- p_x^2 /2mkT\right) dp_x }{\int_{-\infty}^{\infty} \exp\left( -p_x^2 /2mkT \right) dp_x } \right) \nonumber \\
& = \frac{1}{2} kT  \nonumber
\end{align}
Di mana telah dilakukan substitusi $p_x^2 / 2mk T  = u^2 $ \newline
Untuk harmonik osilator, di mana energinya dinyatakan sebagai
\begin{align}
\epsilon_x = \frac{p_x^2}{2m } + \frac{1}{2} \mu x^2 
\end{align}
maka energi rata-ratanya adalah:
\begin{align}
\overline{\epsilon_x} & = \frac{\int_\Gamma \left \lbrace p_x^2 /2m + \frac{1}{2} \mu x  \right \rbrace e^{-\epsilon/kT}d \Gamma}{\int_\Gamma e^{- \epsilon /kT} d \Gamma } \nonumber \\
& = \frac{\int_{-\infty}^{\infty} \int_{-\infty}^{\infty}  \left \lbrace   \frac{p_x^2  }{2m} + \frac{1}{2} \mu x^2\right \rbrace }{\int_{-\infty}^{\infty} \int_{-\infty}^{\infty} } \nonumber \\
& \cdot \frac{ \exp \left( - \left[  \frac{p_x^2  }{2m} + \frac{1}{2} \mu x^2 \right ] /kT \right) }{ \exp \left( - \left[  \frac{p_x^2  }{2m} + \frac{1}{2} \mu x^2 \right ] /kT \right) } \nonumber \\
& \hspace{0.3 cm }\cdot \frac{ dx \, dp_x}{dx \, dp_x}
\end{align}
Jika dilakukan substitusi $p_x^2 /2m  = r^2 \sin^2 \theta $ dan $\frac{1}{2} \mu x^2 = r^2 \cos^2 \theta $ maka diperoleh
\begin{align}
\overline{\epsilon_x} &= \frac{\int_{0}^{2 \pi } d\theta \int_{0}^{\infty} e^{-r^2 /kT} r^3 \, dr }{\int_{0}^{2 \pi} d \theta \int_{0}^{\infty} e^{- r^2 /kT} r \, dr }
\nonumber \\
& = kT \nonumber
\end{align} \newline
\textbf{Statistik Semiklasik} \newline
Secara klasik, jumlah keadaan energi dalam rentang $\epsilon $ sampai $\epsilon + d \epsilon $ dinyatakan oleh:
\begin{align}
g(\epsilon ) d\epsilon = BV 2\pi (2m )^{3/2 }\epsilon^{1/2} d\epsilon  
\end{align}
Sehingga fungsi partisi menjadi
\begin{align}
Z& = \sum_s g_s e^{-\epsilon_s /kT} \nonumber \\&  \equiv \int_{0}^{\infty} e^{-\epsilon/kT } g(\epsilon ) d \epsilon \nonumber  \\
& = 2\pi BV (2m)^{3/2 } \cdot \nonumber \\
& \hspace{0.2 cm } \int_{0}^{\infty} \epsilon^{1/2} e^{-\epsilon /kT} d \epsilon  \nonumber \\ & = BV (2 \pi mkT)^{3/2 } 
\end{align}
yang nantinya akan diperoleh
\begin{align}
F = - Nk T \ln [BV (2 \pi mkT)^{3/2}]  
\end{align} 
dan 
\begin{align}
S &= - \left \lbrace  \frac{\partial F}{\partial T}\right \rbrace  \nonumber  \\
& =\boxed{ Nk \ln [BV (2 \pi mkT )^{3/2 }]+  \frac{3}{2} Nk  }
\end{align}
Dengan persamaan ini akan terdapat kenaikan entropi sebesar $2 Nk \ln 2$ pada pencampuran 2 volume gas.
Fakta ini menuntun pada perumusan statistik semiklasik, di mana bobot konfigurasi dinyatakan oleh 
\begin{align}
W_\mathrm{MB} = \prod_s  \frac{g_s^{n_s}}{n_s !} \label{boltzmann-semiklasik}
\end{align}
\begin{align}
W_\mathrm{BE} = \prod_s \frac{(n_s + g_s -1)!}{n_s ! (g_s - 1)!} 
\end{align}
\begin{align}
W_\mathrm{FD} = \prod_s \frac{g_s! }{n_s ! (g_s - n_s )!} 
\end{align}
Pada limit klasik, yakni $g_s \gg n_s \gg 1$ maka dengan aproksimasi Stirling diperoleh:
\begin{align}
\ln W_\mathrm{MB}& = \sum_s (n_s \ln g_s  - n_s \ln n_s
 \nonumber \\ 
 &\hspace{1 cm } 
 + n_s ) \nonumber \\
& = \sum_s \left( n_s \ln \frac{g_s }{n_s } + n_s  \right)
\end{align}
\begin{align}
\ln W_\mathrm{BE} &\simeq \sum_s [ (n_s + g_s )\ln (n_s + g_s )\nonumber \\ 
&  \hspace{0.3 cm } - n_s \ln n_s - g_s \ln g_s ] \nonumber \\
&  = \prod_s \left [ n_s \ln \left \lbrace  \frac{n_s + g_s }{n_s }\right \rbrace + \right. \nonumber \\
& \hspace{0.2 cm } \left. g_s \ln \left \lbrace  \frac{n_s + g_s}{g_s }\right \rbrace \right ] \nonumber \\
& \simeq \prod_s \left( n_s \ln \frac{g_s }{n_s } + n_s \right)
\end{align}
di mana telah  digunakan hampiran $n_s + g_s - 1 \simeq (n_s + g_s ) \frac{g_s + n_s }{n_s } \simeq \frac{g_s }{n_s }$. Dan $\ln \left \lbrace \frac{n_s + g_s }{g_s} \right \rbrace  = \ln \left \lbrace  1 + \frac{n_s }{g_s } 
\right \rbrace \simeq \frac{n_s }{g_s }$ \newline  Kemudian
\begin{align}
\ln W_\mathrm{FD}& = \sum_s [ g_s \ln g_s - n_s \ln n_s  \nonumber \\
& \hspace{0.2 cm } - (g_s - n_s )\ln (g_s -n_s) ] \nonumber \\
& = \sum_s \left [ n_s \ln \left \lbrace  \frac{g_s - n_s }{n_s }\right \rbrace   - \right. \nonumber
 \\
& \hspace{0.2 cm } \left. g_s \ln \left \lbrace \frac{g_s - n_s }{g_s } \right \rbrace \right ] \nonumber \\
& \simeq \sum_s \left( n_s \ln \frac{g_s }{n_s } + n_s \right) 
\end{align}
Entropi kemudian dinyatakan oleh 
\begin{align}
S = Nk \ln \frac{Z}{N} + \frac{E}{T} + Nk 
\end{align}
Sementara pernyataannya dalam kuantum mekanik adalah:
\begin{align}
Z = \frac{V}{h^3} (2 \pi m kT)
^{3/2 }
\end{align}
dengan demikian entropi untuk sistem semi-klasik dinyatakan oleh 
\begin{align}
S = Nk \left \lbrace \ln \left [ \frac{V (2 \pi m kT )^{3/2} }{Nh^3} \right ] +
  \frac{5}{2} \right \rbrace  \nonumber
\end{align}
energi bebas helmoltz kemudian dinyatakan sebagai 
\begin{align}
F = - kT \ln \frac{Z^N}{N!} 
\end{align}
dengan $\boldsymbol{Z} = Z^N /N!$ menyatakan fungsi partisi total untuk sistem semi-klasik. \newline
\textbf{Ensembel Kanonik} \newline
Ensembel Kanonik adalah gabungan asembli (konfigurasi sistem/partikel) dengan temperatur yang sama. Jadi asembli dapat dipandang sebagai asembli dari asembli, atau observasi-observasi terhadap  asembli sepanjang evolusinya terhadap waktu.\newline
Karena ensembel merupakan asembli dari asembli, maka fungsi partisi pada asembli dapat pula digeneralisasi ke ensembel yakni:
\begin{align}
\boldsymbol{Z} = \sum_i e^{- E_i /kT}
\end{align}
dengan 
\begin{align}
p_i &= p(0) e^{-E_i /k T}  = \frac{e^{-E_i /kT}}{\boldsymbol{Z}}   \label{peluang-kanonik}
\end{align}
menyatakan peluang asembli pada ensembel untuk berada pada keadaan ke-$i$ sementara 
p(0) menyatakan fungsi dari temperatur yakni:
\begin{align}
p(0) = \frac{1}{\sum_i e^{- E_i /kT}}
\end{align}
Jadi
\begin{align}
\sum_i p_i = 1 
\end{align}
Di sini fungsi partisi analog dengan fungsi partisi total dari asembli pada ensembel. Kemudian energi bebas Helmoltz dinyatakan oleh
\begin{align}
F = - kT \ln \boldsymbol{Z}
\end{align}
\newline
\textbf{Fluktuasi energi pada Ensembel Kanonik} \newline
Penyimpangan energi dari nilai rata-rata dinyatakan oleh
\begin{align}
\delta E = E - \overline{ E} 
\end{align}
sehingga 
\begin{align}
\overline{(\delta E)^2 } &= \overline{(E - \overline{E} )^2 }\nonumber \\
& = \overline{( E^2 - 2 E \cdot \overline{E} + \overline{E}^2) } \nonumber \\
& = \overline{E^2} - \overline{E}^2 
\end{align}
dengan 
\begin{align}
\overline{E} = \sum_i p_i E_i  =  \frac{1}{\boldsymbol{Z}} \frac{\partial \boldsymbol{Z}}{\partial \beta} 
\end{align}
dan 
\begin{align}
\overline{ E^2} = \sum_i p_i E_i^2  = \frac{1}{\boldsymbol{Z}} \frac{\partial^2 \boldsymbol{Z}}{\partial \beta^2} 
\end{align}
maka 
\begin{align}
\overline{(\delta E)^2}& = \frac{\partial \overline{E}}{\partial \beta } = kT^2\frac{\partial \overline{E}}{\partial T} \nonumber \\
& = C_v k T^2  
\end{align}
Dengan demikian fraksi fluktuasi energi dapat dinyatakan sebagai
\begin{align}
\mathcal{F} & = \left \lbrace \frac{\overline{(\delta E)^2}}{\overline{E}^2} \right \rbrace^{1/2} \nonumber \\
& = \left \lbrace \frac{kT^2 C_v }{\overline{E}^2}\right \rbrace^{1/2}
\end{align}
dengan $C_v = (3/2) Nk$ dan $\overline{E} = (3/2)NkT$ maka 
\begin{align}
\mathcal{F} &= \left \lbrace \frac{\frac{3}{2} N k^2 T^2 }{\left( \frac{3}{2} Nk T\right)^2} \right \rbrace^{1/2} \nonumber \\
& = \left( \frac{3}{2} N\right)^{- 1/2 }
\end{align}
\newline
\textbf{Enesembel Grand Kanonik}\newline
Fungsi partisi grand-kanonik dinyatakan oleh persamaan 
\begin{align}
\boldsymbol{\mathcal{Z} }= \sum_i e^{(\mu N - E)_i / kT }  \label{partisi grand kanonik}
\end{align}
dengan 
\begin{align}
p_i &=  e^{-[ pV + (\mu N_i - E_i )/k T}
 \label{peluang grand kanonik}
\end{align}
di mana 
\begin{align}
&\sum_i p_i  = 1  \\
& \text{atau } \\
 &  e^{- pV /kT} \sum_i e^{(\mu N_i  - E_i )/kT} = 1
\end{align}
atau 
\begin{align}
  e^{- pV/kT}& = \frac{1}{\sum_i e^(\mu N_i - E_i )/kT} \nonumber  \\
&  = \frac{1}{\boldsymbol{\mathcal{Z}} } \label{identitas grand kanonik}
\end{align}
Sehingga 
\begin{align}
p_i = \frac{e^{(\mu N_i - E_i )/kT}}{\boldsymbol{\mathcal{Z}}} 
\end{align}
analog dengan pers.\ref{peluang-kanonik} untuk ensembel kanonik.\newline 
Kemudian 
\begin{align}
\overline{N}& =  \sum_i p_i N_i \nonumber \\
 & = \sum_i \frac{N_i e^{(\mu N_i - E_i )/kT}}{\boldsymbol{\mathcal{Z}}} \nonumber \\
& = \frac{kT}{\boldsymbol{\mathcal{Z}}} \left \lbrace \frac{\partial \boldsymbol{\mathcal{Z}}}{\partial \mu }\right \rbrace \nonumber \\
& = kT \left \lbrace \frac{\partial \ln \boldsymbol{\mathcal{Z}}}{\partial \mu} \right \rbrace_{V,T} 
  \label{jumlah rata-rata partikel}
\end{align}
dan 
\begin{align}
\overline{N^2} = \frac{(kT)^2}{\boldsymbol{\mathcal{Z}}} \left \lbrace 
\frac{\partial^2 \boldsymbol{ \mathcal{Z}}}{\partial \mu^2 } \right \rbrace_{V,T} \label{jumlah rata-rata kuadrat partikel}  
\end{align}
Dari pers.\ref{identitas grand kanonik} dapat pula diperoleh
\begin{align}
(pV) = kT \ln \boldsymbol{\mathcal{Z}} 
\end{align}
sehingga jumlah rata-rata partikel pada pers.\ref{jumlah rata-rata partikel} dapat dinyatakan ke dalam
\begin{align}
\overline{ N} = \left \lbrace \frac{\partial (pV)}{\partial \mu } \right \rbrace_{T,V} 
\end{align}
Perhatikan
\begin{align}
d(pV) = p\,  dV  + S \, d T + \overline{N} \, d \mu 
\end{align}
demikian pula 
\begin{align}
p & = \left \lbrace \frac{\partial (pV)}{\partial V} \right \rbrace_{T, \mu}
\end{align}
\begin{align}
S& = \left \lbrace  \frac{\partial (pV)}{\partial T}\right \rbrace_{V, \mu}
\end{align}
Untuk distribusi Bose-einstein, pers. \ref{jumlah rata-rata partikel} akan menghasilkan
\begin{align}
\overline{N} &= kT \frac{\partial}{\partial \mu } \left \lbrace \ln \prod_j [1 - \right. \nonumber \\ 
& \hspace{0.2 cm } \left. e^{(\mu - \epsilon_j)kT}]^{-1} \right \rbrace_{V,T} \nonumber \\
& = - kT \sum_j \left \lbrace \frac{\partial}{\partial \mu} \right. \nonumber \\
& \hspace{0.2 cm} \left. \ln[1 - e^{(\mu - \epsilon_j )/kT}] \right \rbrace_{V,T} \nonumber \\
& = \sum_j \frac{1}{e^{(\epsilon_j - \mu )/kT} - 1} \nonumber \\
& = \sum_j \overline{n_j }
\end{align}
Dengan cara yang sama untuk distribusi fermi-dirac diperoleh
\begin{align}
\overline{N} &  = \sum_j \overline{n_j } \nonumber \\
& = \sum_j \frac{1}{e^{(\epsilon_j - \mu )/kT} + 1}
\end{align}
\newline
\textbf{Fluktuasi Jumlah Partikel dalam ensembel grand kanonik} \newline
Dari pers. \ref{jumlah rata-rata partikel} dan pers. \ref{jumlah rata-rata kuadrat partikel} dapat diturunkan
\begin{align}
& \frac{\partial}{\partial \mu } \left \lbrace  \frac{1}{\boldsymbol{ \mathcal{Z}}} \left( \frac{\partial \boldsymbol{\mathcal{Z}}}{\partial \mu }\right)\right  \rbrace_{V,T}\nonumber \\ &  = \left\lbrace  \frac{1}{\boldsymbol{\mathcal{Z}}} \frac{\partial^2 \boldsymbol{\mathcal{Z}}}{\partial \mu^2}  - \frac{1}{\boldsymbol{\mathcal{Z}}^2} \left(\frac{\partial \boldsymbol{\mathcal{Z}}}{\partial \mu }\right)^2\right \rbrace_{V,T}
\end{align}
di mana telah disebutkan
\begin{align}
\frac{N}{kT} = \frac{1}{\boldsymbol{\mathcal{Z}}} \left \lbrace \frac{\partial \boldsymbol{\mathcal{Z}}}{\partial \mu } \right \rbrace_{V,T}
\end{align}
ini akhirnya menghasilkan
\begin{align}
&\frac{1}{(kT)} \left \lbrace \frac{\partial \overline{N}}{\partial \mu }\right \rbrace_{V,T}  \nonumber \\
 &  = \frac{1}{(kT)^2} (\overline{N^2} - \overline{N}^2 )
\end{align}
atau 
\begin{align}
\overline{(\delta N)^2 } = kT \left \lbrace  \frac{\partial \overline{N}}{\partial \mu}\right \rbrace_{V,T} 
\end{align}
 \newline
\textbf{Penurunan  persamaan \ref{persamaan gelombang em}} \newline
Persamaan schrodinger tak bergantung waktu dinyatakan oleh:
\begin{align}
\frac{\partial^2 \psi}{\partial x^2} + \frac{\partial^2 \psi}{\partial y^2} + \frac{\partial^2 \psi}{\partial z^2} + \frac{8 \pi^2 m E}{h^2} \psi  = 0 
\end{align}
Untuk partikel bebas, solusinya dinyatakan oleh
\begin{align}
\psi  = \psi_0 e^{j(k_x x + k_y y + k_z z)} \label{solusi} 
\end{align}
di mana $k_x  = 2 \pi / \lambda_x $, $k_y = 2 \pi / \lambda_y$, $2 \pi / \lambda_z$ dan $(p_x^2 + p_y^2 + p_z^2 ) = 2 m E$. \newline
Jika partikel/sistem dianggap terperangkap dalam kotak yang sisi-sisinya diberikan oleh $L_x$, $L_y$, $L_z$ maka dengan menerapkan sarat batas terhadap solusi (pers. \ref{solusi}), akan diperoleh
\begin{align}
\left. 
\begin{array}{l}
n_x \frac{\lambda_x}{2} = L_x \\
n_y \frac{\lambda_y}{2} = L_y \\
n_z \frac{\lambda_z}{2} = L_z   
\end{array} \right \rbrace \label{jumlah setengah panjang gelombang}
\end{align}
yang menyatakan jumlah "setengah panjang gelombang" pada masing-masing arah. Mengingat $\lambda_x = 2 L_x / n_x $ maka  pers. (\ref{jumlah setengah panjang gelombang}) dapat untuk arah-x dapat dinyatakan menjadi 
\begin{align}
n_x = \frac{2 L_x p_x }{h} 
\end{align}
demikian pula untuk arah yang lain. Sementara itu diperoleh pula 
\begin{align}
(n_x + d n_x ) = \frac{2\, L_x (p_x + d p_x)}{h}
\end{align}
atau 
\begin{align}
d n_x = \frac{2 L_x d p_x }{h}
\end{align}
demikian pula untuk arah yang lain. \newline
Dengan demikian diperoleh
\begin{align}
dn_s& = d n_x \, d n_y \, d n_z \nonumber \\
& = \frac{8 L_x L_y L_z  \, d p_z d p_y d p_z}{h^3}  \label{seluruh kuadran}
\end{align}
Karena dalam satu sumbu, ada dua keadaan momentum yang dibolehkan, yakni  ($p_x$ dan $-p_x$), maka nilai aktual dari jumlah keadaan
 $d \mathfrak{n}$ yang memiliki nilai momentum dalam rentang $+ \, dp_x$, $+\, d p_y$, $+ \, d p_z$ adalah 1/8 dari nilai  pers.\ref{seluruh kuadran}, yakni kuadran pertama koordinat rectilinear  atau
 \begin{align}
 d \mathfrak{n} = \frac{L_x L_y L_z \, d p_z d p_y d p_z }{h^3} \label{1/8-kuadran}
 \end{align}
 Jika $L_x L_y L_z $ menyatakan volume $V$, maka
pers.\ref{1/8-kuadran} dapat ditransformasikan ke dalam pernyataan pada koordinat bola, yakni 
\begin{align}
d \mathfrak{n} =  \frac{V \cdot 4 \pi p^2  \, dp }{h^3}  = \frac{\Delta \Gamma}{h^3}
\end{align}
Jika pernyataan ini dinyatakan dalam panjang helombang $\lambda  = h/p$, maka diperoleh 
\begin{align}
d \mathfrak{ n} = g (\lambda ) d \lambda = \frac{4 \pi }{\lambda^4} d \lambda 
\end{align}
\textbf{Ekspansi Sommerfeld}
Aproksimasi Sommerfeld digunakan untuk menghitung pernyataan integral 
\begin{align}
N & = \int_{0}^{\infty} g(\epsilon) \overline{n}_\mathrm{FD} \, d \epsilon \nonumber \\
& = g_0 \int_{0}^{\infty} \epsilon^{1/2} \overline{n}_\mathrm{FD} \, d\epsilon 
\end{align}
dengan $\overline{n}_\mathrm{FD}$ menyatakan fungsi distribusi fermi-dirac (dalam buku pointon dinyatakan sebagai  $f(\epsilon)$). 
Karen daerah yan g ditinjau hanya disekitar $\epsilon = \mu $,maka integralnya dapat dinyatakan ke dalam integral parsial
\begin{align}
N  & = 
 \frac{2}{3} g_0 \epsilon^{3/2} \overline{n}_\mathrm{FD} (\epsilon)  \vline_{0}^{\infty} +  \nonumber \\
 &  \frac{2}{3} g_0 \int_0^\infty \epsilon^{3/2} \left( - \frac{d \overline{n}_\mathrm{FD}}{d \epsilon }\right) d \epsilon 
\end{align}  
suku pertama akan habis pada kedua batas integral sementara suku kedua dinyatakan kembali melalui 
\begin{align}
- \frac{d \overline{n}_\mathrm{FD}}{d \epsilon } & = - \frac{d}{d \epsilon } (e^{(\epsilon - \mu )/kT }+ 1)^{-1} \nonumber \\
& = \frac{1}{kT} \frac{e^x}{(e^x + 1) ^2} 
\end{align}
dengan $x = (\epsilon - \mu )/ kT$. \newline Jadi 
\begin{align}
N& = \frac{2}{3} g_0 \int_{0}^{\infty} \frac{1}{kT} \frac{e^x}{(e^x +1)^2} \epsilon^{3/2} \, d \epsilon  \nonumber \\
& = \frac{2}{3} g_0 \int_{-\mu/kT}^{\infty} \frac{e^x}{(e^x + 1)^2} \epsilon^{3/2}  \, d \epsilon
\end{align}
Ada dua pendekatan yang dilakukan, yang pertama adalah
dengan melakukan uraian Taylor terhadap $\epsilon^{3/2} $ di sekitar $\epsilon = \mu $  dan hanya mengambil beberapa suku pertama. Sementara yang kedua adalah meng-ekstends batas bawah integralnya sampai $- \infty$ agar mudah dalam mnipulasi matematis selanjutnya dan ini sama sekali tidak berpengaruh pada fungsi secara keseluruhan mengingat pada daerh negatif, fungsinya nol.
\newline Dengan demikian diperoleh
\begin{align}
\epsilon^{3/2}& = \mu^{3/2} + (\epsilon - \mu  ) \frac{d}{d \epsilon} \epsilon^{3/2} \vline_{\epsilon = \mu } \nonumber \\
&+ \frac{1}{2} (\epsilon - \mu )^2  \frac{d^2}{d \epsilon^2} \epsilon^{3/2} \vline_{\epsilon = \mu } + \cdots \nonumber \\ 
& = \mu^{3/2} + \frac{3}{2} (\epsilon -\mu )\mu^{1/2} + \nonumber \\
& \frac{3}{8} (\epsilon - \mu )^2 \mu^{1/2} + \cdots  
 \end{align}
sehingga 
%\columnbreak
%\pagebreak
\begin{align}
N & = \frac{2}{3} g_0 \int_{-\infty}^{\infty} \frac{e^x}{(e^x + 1)^2} \left [ \mu^{ 3/2} + \right. \nonumber \\
 & \frac{3}{2} (\epsilon - \mu )\mu^{1/2}  + \frac{3}{8} (\epsilon - \mu)^2 \mu^{1/2}\nonumber \\
 & \hspace{0.2 cm } + \cdots 
\end{align}
Integrasi selanjutnya dapat dilakukan pada masing-masing suku yakni untuk suku pertama adalah
\begin{align}
&\int_{- \infty}^{\infty} \frac{e^x}{(e^x + 1)^2} \, dx  = 
\nonumber \\
& \int_{- \infty}^{\infty} - \frac{d \overline{n}_\mathrm{FD}}{d \epsilon} \, d\epsilon \nonumber \\
& = \overline{n}_\mathrm{FD} (- \infty) - \overline{ n}_\mathrm{FD} (\infty)  =\nonumber \\
& \hspace{0.2 cm } 1- 0 =1
 \end{align}
 Untuk suku kedua 
 \begin{align}
& \int_{-\infty}^{\infty} \frac{x e^x }{(e^x + 1)^2}  \, dx =   \nonumber \\
 & \int_{-\infty}^{\infty} \frac{x}{(e^x + 1)(1+ e^{-x})} \, dx  = 0. 
 \end{align}
 mengingat pernyataan tersebut merupakan fungsi ganjil dari $x$. \newline
 Suku ketiga dapat diintegrasikan secara parsial secara berurutan yang natinya akan menghasilkan 
 \begin{align}
 \int_{- \infty}^{\infty} \frac{x^2 e^x }{(e^x +1)^2 } \,  dx = \frac{\pi^2}{3} 
 \end{align}
 Dengan mengumpulkan hasil-hasil tersebut, maka nilai $N$ selanjutnya dapat dituliskan menjadi
 \begin{align}
 N & = \frac{2}{3} g_0 \mu^{3/2} + \frac{1}{4} g_0 (kT)^2 \mu^{-1/2}\nonumber \\
 & \hspace{0.1 cm }  \cdot \frac{\pi^2} {3} + \cdots \nonumber \\
 & = N \left(\frac{\mu}{\epsilon_\mathrm{F}}\right)^{3/2 } + N \frac{\pi^2}{8} \frac{(kT) ^2}{\epsilon_\mathrm{F}^{3/2} \mu^{1/2}} \nonumber \\
 & \hspace{0.3 cm } + \cdots
 \end{align}
 Di mana telah dilakukan substitusi untuk $g_0  = 3N / 2 \epsilon_\mathrm{F}^{3/2}$. Dengan membagi kedua ruas dengan $N$ diperoleh
 \begin{align}
 \frac{\mu}{\epsilon_\mathrm{F}}  &=  \left [  1 -  \frac{\pi^2}{8} \left(\frac{kT}{\epsilon_\mathrm{F}}\right)^2 + \cdots\right]^{2/3} \nonumber \\
 & = 1- \frac{\pi^2}{12} \left(\frac{kT}{\epsilon_\mathrm{F}}\right)^2  + \cdots \label{potensial kimia}
 \end{align} 
 yang menunjukkan  potensial kimia $\mu$ akan naik secara berangsur-angsur seiring naiknya $T$. \newline
 Hasil ini juga dapat digunakan untuk menhitung integral untuk nilai total energi yakni
 \begin{align}
 U &= \int_{0}^{\infty} \epsilon g(\epsilon) \overline{n}_\mathrm{FD} (e\epsilon) d \epsilon \nonumber \\
 & = \int_{0}^{\infty} \epsilon g(\epsilon) \frac{1}{e^{(\epsilon - \mu)/kT}+1}  \, d\epsilon 
 \end{align}
yakni 
\begin{align}
U& =\frac{3}{5} N \frac{\mu^{5/2}}{\epsilon_\mathrm{F}^{3/2}} + \nonumber \\
& \hspace{0.2 cm} \frac{3\pi^2 }{8} N \frac{(kT)^2}{\epsilon_\mathrm{F}} + \cdots
\end{align}
Di mana dengan memasukkan pers. \ref{potensial kimia} diperoleh
\begin{align}
U&  = \frac{3}{5} N \epsilon_\mathrm{F} + \frac{\pi^2}{4} N \frac{(kT)^2}{\epsilon_\mathrm{F}} + \cdots
\end{align}
\textbf{Kondensasi Bose-Einstein} \newline
Kondensasi bose-eintein adalah kondensasi di mana sebagian besar partikel penyusun gas boson akan mengalami kondensasi/pendinginan yakni menempati keadaan energi dasar (ground state). \newline
Jumlah boson yang menempati keadaan dasar dinyatakan oleh:
\begin{align}
N_0 & = \frac{1}{e^{(\epsilon_0 - \mu )/kT} - 1} \label{jumlah partikel dalam ground state}
\end{align}
Untuk $T =0$, nilai $N_0$ cukup besar yang bisa terpenuhi jika peneyebut  pada pers. \ref{jumlah partikel dalam ground state} cukup kecil. Dengan demikian potensial kimia $\mu$ hanya berbeda cukup kecil dengan $\epsilon$.  Dan $kT \gg \epsilon_0$. Jumlah total boson tentunya dinyatakan oleh
\begin{align}
\boxed{N = \sum_{\mathrm{semua \hspace{0.1 cm} s}} \frac{1}{e^{(\epsilon_s - \mu)/ kT} - 1} }
\end{align}
karena masing-masing suku pada penjumlahan cukup kecil ($kT \gg \epsilon_0 \approx \mu$ ), dapat dilakukan transformasi ke dalam bentuk integral, yakni:
\begin{align}
N = \int_{0}^{\infty} g(\epsilon ) \frac{1}{e^{(\epsilon_0 - \mu)/kT} -1 } \, d\epsilon
\end{align}
dengan $g(\epsilon)$ dinyatakan oleh persamaan \ref{density of states}. Dengan demikian 
\begin{align}
N &= \frac{2}{\sqrt{\pi}} \left(\frac{2 \pi m}{h^2}\right)^{3/2} V \nonumber \\
& \hspace{0.2 cm } \cdot \int_{0}^{\infty} \frac{\sqrt{\epsilon} \, d \epsilon}{e^{\epsilon/kT} -1} \nonumber \\
& = \frac{2}{\sqrt{\pi}} \left( \frac{2 \pi mk T}{h^2}\right)^{3/2} V \nonumber \\ & \hspace{0.2 cm} \cdot \int_{0}^{\infty} \frac{\sqrt{x} \, dx}{e^x - 1}
\end{align}
Integral terhadap $x$ nilainya adalah $2.315$ sehingga dengan mengkomninasikan dengan $2/ \sqrt{\pi}$ akan menghasilkan formula 
\begin{align}
N = 2.162 \left( \frac{2 \pi mkT}{h^2}\right)^{3/2} V 
\end{align}
Karena jumlah boson $N$ tidak bergantung pada temperatur $T$ makan hanya satu nilai temperatur di mana persamaan di atas benar, yakni $T = T_c $, sehingga diperoleh
\begin{align}
N = 2.162 \left( \frac{2 \pi mkT_c}{h^2}\right)^{3/2} V 
\end{align}
atau 
\begin{align}
kT_c = 0.527 \left( \frac{h^2}{2 \pi m}\right) \left( \frac{N}{V}\right)^{2/3}
\end{align}
sementara jumlah boson dalam kedaan tereksitasi, dinyatakan oleh
\begin{align}
N_\mathrm{excited} &= 2.612 \left( \frac{2 \pi m kT}{h^2}\right)^{3/2} V \nonumber \\
& \hspace{0.2 cm } (\text{untuk } T < T_c )
\end{align}
Dengan demikian 
\begin{align}
N_\mathrm{excited} &  = \left( \frac{T}{T_c }\right)^{3/2} N  \nonumber \\
& \hspace{0.3 cm } (\text{untuk }) (T < T_c)
\end{align}
dan 
\begin{align}
N_0 &= N - N_\mathrm{excited} \nonumber \\
& = \left [  1 - \left( \frac{T}{T_c }\right)^{3/2} \right ] N  \nonumber \\
& \hspace{ 0.5 cm } (\text{untuk } T < T_c)
\end{align}
\textbf{Contoh penentuan fungsi partisi dan peluang keadaan suatu partikel} \newline 
1. \textbf{ Dipol elementer} dengan energi "up states" $-\mu B$ dan "down states"$+\mu B$, maka fungsi partisinya adalah
\newcommand{\pindah}{\nonumber \\}
\newcommand{\sama}{& =}
\begin{align}
Z &= \sum_s e^{-\beta E(s)} \pindah 
\sama e^{+ \beta \mu B} +  e^{-\beta \mu B} \pindah 
\sama 2 \cosh (\beta \mu B)
\end{align}
Peluang dipole untuk berada dalam "up states" adalah
\begin{align}
\mathcal{P}_\uparrow = \frac{e^{+\beta \mu B}}{Z} = \frac{e^{+ \beta \mu B}}{2 \cosh (\beta \mu B)}
\end{align}
Peluang dipole untuk berada dalam "down states" adalah
\begin{align}
\mathcal{P}_\downarrow = \frac{e^{-\beta \mu B}}{Z} =  \frac{e^{- \beta \mu B}}{2 \cosh (\beta \mu B)}
\end{align}
Dengan demikian energi rata-rata adalah
\begin{align}
\overline{E}& = \sum_s E(s) \mathcal{P}(s) \nonumber \\ 
& = (- \mu B )\mathcal{P}_\uparrow + (+ \mu B)\mathcal{P}_\downarrow  \nonumber \\
& = - \mu B \frac{e^{\beta \mu B} - e^{- \beta \mu B}}{2 \cosh (\beta \mu B)}  \nonumber \\
& = - \mu B \tanh (\beta \mu B) 
\end{align}
Sementara total energi dinyatakan oleh
\begin{align}
U = - N \mu B \tanh(\beta \mu B)
\end{align}
Momen magnetik dipole kemudian dinyatakan oleh
\begin{align}
\overline{\mu_z } &= \sum_s \mu_s (s) \mathcal{P}(s)  \nonumber \\
& = (+ \mu ) \mathcal{P}_\uparrow + (\mu ) \mathcal{P}_\downarrow \nonumber \\
& = \mu \tanh(\beta \mu B) 
\end{align}
Sementara total magnetisasi dinyatakan oleh 
\begin{align}
M = N \overline{\mu_z} = N \mu \tanh (\beta \mu B)
\end{align}
Perhatikan antara energi dipol dan momen Magnetik berbeda tanda.\newline 
2. \textbf{Pembuktian Teorema} equipartisi dengan menggunakan fungsi partisi. Jika energi dinyatakan dalam 
\begin{align}
E(q) = c q^2 
\end{align}
anggap yang ditinjau hanya dalam satu derajat kebebasan, sehingga fungsi partisinya dinyatakan dalam
\begin{align}
Z & = \sum_q e^{- \beta E(q)} = \sum_q e^{-\beta cq^2}
\end{align}
jika ditambahkan $\Delta$ di dalam dan di luar persamaan di atas, maka akan diperoleh
\begin{align}
Z& = \frac{1}{\Delta q} \sum_q e^{-\beta c q^2} \nonumber \\
& = \frac{1}{\Delta q} \int_{-\infty}^{\infty} e^{-\beta c q^2} \nonumber \\
& = \frac{1}{\Delta q} \frac{1}{\sqrt{\beta c}} \int_{-\infty }^{\infty} e^{- x^2} \, dx  \nonumber \\
& = \frac{1}{\Delta q} \sqrt{\frac{\pi}{\beta c}}  = C \beta^{-1/2} 
\end{align}
dengan $C = \sqrt{\pi / c } / \Delta q$. Dengan demikian 
\begin{align}
\overline{ E} &= -  \frac{1}{Z} \frac{\partial Z}{\partial \beta } \nonumber \\
& = - \frac{1}{C \beta ^{- 1/2}} \frac{\partial }{\partial \beta } C \beta^{-1/2} \nonumber \\
& = - \frac{1}{C \beta^{-1/2} } (- 1/2) C \beta^{- 3/2} \nonumber \\
& = \frac{1}{2} \beta^{-1}  = \frac{1}{2} kT
\end{align}
3. \textbf{Oilator Harmonik} dalam kuantum memiliki energi dinyatakan dalam 
\begin{align}
\epsilon = (n + 1/2) h \nu 
\end{align}
dengan demikian fungsi partisinya dinyatakan oleh
\begin{align}
Z & = \sum_{n = 0}^{\infty} e^{- (n+ 1/2)h \nu / kT} \nonumber \\
& = e^{- 1/2 \, h \nu / kT} \sum_{n = 0}^{\infty} e^{- n h \nu / kT} \nonumber \\
& = \frac{e^{- 1/2 h \nu /k T}}{(1 - e^{- h\nu /kT})}
\end{align}
dengan demikian 
\begin{align}
\overline{\epsilon} = h \nu \left \lbrace  1/2  + \frac{1}{e^{h\nu /kT} - 1}\right \rbrace 
\end{align}
untuk temperatur tinggi $h\nu /kT \ll 1$ sehingga 
\begin{align}
e^{h\nu /kT} &\simeq 1 + \left \lbrace \frac{h\nu}{kT} \right \rbrace + \frac{1}{2} \left \lbrace \frac{h\nu}{kT} \right \rbrace^{2} 
\end{align}
dengan demikian 
\begin{align}
\overline{\epsilon} & \simeq h\nu \left \lbrace \frac{1}{2} + \frac{1}{\frac{h\nu}{kT} + \frac{1}{2} \left (\frac{h\nu}{kT} \right )^2}  \right \rbrace  \nonumber \\
&\simeq h \nu \left \lbrace \frac{1}{2} + \frac{kT}{h\nu} \left( 1 - \frac{1}{2} \frac{h\nu}{kT}\right) \right \rbrace  \nonumber \\
& = kT
\end{align}
4. Fungsi partisi \textbf{Molekul diatomik} akan dinyatakan oleh
\begin{align}
Z & = Z_t Z_r Z_v Z_e Z_n 
\end{align}
$Z_t $ menyatakan fungsi partisi berkaitan dengan gerak translasi yang dinyatakan oleh
\begin{align}
Z_t = \frac{V}{h^3} (2\pi m kT)^{3/2} 
\end{align}
Untuk gerak rotasi, energinya dinyatakan oleh $\epsilon_j = j(j+1) \frac{h^2} {8 \pi \mathcal{I}}$. Sehingga fungsi partisinya akan menjadi 
\begin{align}
Z_r & = \sum_{j = 0}^{\infty} (2 j + 1) e^{-\epsilon_j /kT} \nonumber \\
& = \sum_{j = 0}^{\infty} (2 j + 1) e^{- j (j+1)K /k T}  
\end{align}
dengan $K =  \frac{h^2} {8 \pi \mathcal{I}}$ \newline
Kemudian fungsi partisi yang berkaitan dengan gerak vibrasi dinyatakan oleh 
\begin{align}
\boxed{Z_\nu = \frac{e^{- (1/2) h\nu /k T}}{1 - e^{-  h\nu /kT}}}
\end{align}
Fungsi partisi elektronik kemudian dinyatakan oleh 
\begin{align}
Z_e &  = g_0 + g_1 e^{- \epsilon_{e1} /kT} + \nonumber \\
& \hspace{0.3 cm } g_2 e^{- \epsilon_{e2}/ kT} + \cdots
\end{align}
Tidak ada kontribusi dari $Z_n $ karena nilainya tidak bergantung pada temperatur.
 \newline
\textbf{Jawaban Soal-Soal} \newline
\textbf{1. soal bab 4 buku pointon} \newline
Fungsi distribusi bose-einstein sendiri menyatakan bahwa jumlah foton yang berada pada tingkat energi  tertentu dinyatakan oleh hubungan  
\begin{align}
n_s = \frac{g_s}{e^{h v_s/kT} - 1} 
\end{align}
Sementara pada foton berlaku $h\nu = h c /\lambda $. Dengan demikian persamaan (i.13) dapat dimasukkan ke dalam persamaan  (i.14)  sebagai jumlah foton, $n_\lambda (\lambda) d \lambda$ dengan panjang gelombang antara $\lambda$ sampai $\lambda + d \lambda $ yakni
\begin{align}
n_\lambda (\lambda) d \lambda = \frac{8 \pi}{\lambda^4} d \lambda \cdot \frac{1}{e^{hc/ k \lambda T} - 1} \label{jumlah foton}
\end{align}
jumlah total foton dalam kotak dapat diperoleh dengan mengintegralkan persamaan (\ref{jumlah foton}) terhadap semua panjang gelombang yang mungkin, yakni 
\begin{align}
N_\mathrm{tot} = \int_{0}^\infty \frac{8 \pi}{\lambda^4}d\lambda \cdot \frac{1}{e^{hc/k \lambda T} - 1}  \label{total foton}
\end{align}
Jika digunakan pemisalan  $\displaystyle t = hc/ k\lambda T$ maka $ \displaystyle dt = \frac{hc}{kT}(-d\lambda / \lambda^2)$ atau $\displaystyle \frac{d\lambda }{\lambda^2} = - \frac{ kT dt}{hc}$ dan $\displaystyle \lambda^2 = \left(\frac{hc}{kT}\right)^2 t^{-2}$ dan batas pengintegralan akan saling terbalik. Jadi persamaan (\ref{total foton}) dapat dituliskan menjadi
\begin{align}
N_\mathrm{tot} & = 8\pi \int_\infty^0 \left(- \frac{kT}{hc} dt \right) \left( \frac{kT }{hc} t\right)^2 \nonumber \\
& \hspace{0.2 cm } \cdot \left( \frac{1}{e^t - 1}\right) \nonumber \\
& = 8\pi \left(\frac{kT}{hc }\right)^3 \int_{0}^{\infty} \frac{t^2}{e^t -1} \label{n_tot}
\end{align}
Persamaan terkahir dapat diselesaikan dengan meninjau hubungan berikut
\begin{align}
\frac{1}{e^t - 1} = \frac{e^{-t}}{1- e^{-t}}  
\end{align}
Dengan mengambil uraian taylor $\displaystyle \frac{1}{1 - x} = \sum_{n=0}^\infty x^n \hspace{0.3 cm} $(untuk $ |x| < 1$) pada suku penyebut maka
\begin{align}
\frac{1}{e^t - 1} = e^{-t}+ (e^{-t} )^2 + (e^{-t})^3 + ...  \label{hasil uraian}
\end{align}
Kemudian tinjau integral berikut 
\begin{align}
& \int_0^\infty t^2 e^{-n t } =  
 \left. - \frac{1}{n} e^{-nt} \cdot t^2 \right \vert_0^\infty + \nonumber 
 \\
& \frac{2}{n}  \left [\left. -\frac{1}{n} e^{-n t} t \right \vert_0^\infty + \frac{1}{n} \int_{0}^{\infty}  e^{-nt} dt \right ]
\nonumber 
\\
& =  \left [ -  \frac{1}{n} e^{-n t} t^2 - \frac{2}{n^2} e^{-nt} t\, - \right.
 \nonumber \\
& \left. \frac{2}{n^3} e^{-nt}  \right]_0^\infty 
\end{align}
Untuk $t$ besar maka $e^t \gg t^2$ jadi $\displaystyle \lim_{t \rightarrow \infty} \frac{t^2}{e^t} = 0$. Sehingga 
\begin{align}
& \left [- \frac{1}{n} e^{-n t} t^2 - \frac{2}{n^2} e^{-nt} t\,  - \right. \nonumber \\
& \left.  \frac{2}{n^3} e^{-nt}  \right]_0^\infty = \frac{2}{n^3} \label{hasil limit}
\end{align}
Jadi dengan memasukkan hasil (\ref{hasil limit}) dan uraian (\ref{hasil uraian}) ke dalam persamaan (\ref{n_tot}) maka diperoleh total jumlah foton pada kotak dalam kesetimbangan termal  yakni:
\begin{align}
\boxed{16 \pi \left(\frac{kT}{hc}\right)^3 \sum_1^\infty \frac{1}{n^3}}
\end{align}
\textbf{2. Penentuan fungsi partisi untuk statistik fermi dirac, bose-einstein, dan statistik klasik terbedakan}
\newline
Misal, $\epsilon _0 = 0 $; $\epsilon_1 = \epsilon $;$ \epsilon_2 = 2 \epsilon$, $N = 2; g_0 = g_1 = 2; g_2 = 1 $. Maka untuk statistik fermi-dirac akan diperoleh tabulasi sebagai berikut: \newline
\begin{tabular}{|c | c| }
 \cline{1-1} \\ \hline
 &  \\ \hline 
x & x  \\
\hline
\multicolumn{2}{|c|}{$\epsilon = 0$} \\ \hline
\end{tabular}
\begin{tabular}{|c | c| }
 \cline{1-1}  \\ \hline
x &  \\ \hline 
x &   \\
\hline
\multicolumn{2}{|c|}{$\epsilon = 1$} \\ \hline
\end{tabular}
\begin{tabular}{|c | c| }
 \cline{1-1}  \\ \hline
 & x \\ \hline 
x &   \\
\hline
\multicolumn{2}{|c|}{$\epsilon = 1$} \\ \hline
\end{tabular}
\newline 
\begin{tabular}{|c | c| }
 \cline{1-1}  \\ \hline
x &  \\ \hline 
 & x  \\
\hline
\multicolumn{2}{|c|}{$\epsilon = 1$} \\ \hline
\end{tabular}
\begin{tabular}{|c | c| }
 \cline{1-1}  \\ \hline
 & x \\ \hline 
 & x  \\
\hline
\multicolumn{2}{|c|}{$\epsilon = 1$} \\ \hline
\end{tabular}
\begin{tabular}{|c | c| }
 \cline{1-1} x \\ \hline
 &  \\ \hline 
x &   \\
\hline
\multicolumn{2}{|c|}{$\epsilon = 2$} \\ \hline
\end{tabular} \newline
\begin{tabular}{|c | c| }
 \cline{1-1} x \\ \hline
 &  \\ \hline 
 & x  \\
\hline
\multicolumn{2}{|c|}{$\epsilon = 2$} \\ \hline
\end{tabular}
\begin{tabular}{|c | c| }
 \cline{1-1}  \\ \hline
x &  x \\ \hline 
 &   \\
\hline
\multicolumn{2}{|c|}{$\epsilon = 2$} \\ \hline
\end{tabular}
\begin{tabular}{|c | c| }
 \cline{1-1} x \\ \hline
x &  \\ \hline 
 &   \\
\hline
\multicolumn{2}{|c|}{$\epsilon = 3$} \\ \hline
\end{tabular} \newline
\begin{tabular}{|c | c| }
 \cline{1-1} x \\ \hline
 & x \\ \hline 
 &   \\
\hline
\multicolumn{2}{|c|}{$\epsilon = 3$} \\ \hline
\end{tabular} \newline
Dengan demikian fungsi partisinya adalah:
\begin{align}
\boldsymbol{Z}& =\sum_i W_i e^{- \epsilon_i /kT} \nonumber \\
& = e^0 + 4 e^{- \epsilon/ kT} + 3 e^{-2\epsilon /kT} + 2 e^{-3\epsilon /kT}  
\end{align}
sehingga energi rata-ratanya adalah
\begin{align}
\overline{E}& = \frac{kT^2}{\boldsymbol{Z}} \frac{\partial \boldsymbol{Z}}{\partial T} \nonumber \\
& = \frac{2 \epsilon e^{- \epsilon /kT}}{\boldsymbol{Z}} (2 + 3 e^{- \epsilon /k T }+ \nonumber \\
& \hspace{ 0.3 cm }3 e^{- 2 \epsilon /kT})
\end{align}
Untuk statistik bose-einstein, tabulasinya adalah sebagai berikut 
\begin{tabular}{|c | c| }
 \cline{1-1}  \\ \hline
 &  \\ \hline 
xx &   \\
\hline
\multicolumn{2}{|c|}{$\epsilon = 0$} \\ \hline
\end{tabular}
\begin{tabular}{|c | c| }
 \cline{1-1}  \\ \hline
 &  \\ \hline 
 & xx  \\
\hline
\multicolumn{2}{|c|}{$\epsilon = 0$} \\ \hline
\end{tabular}
\begin{tabular}{|c | c| }
 \cline{1-1}  \\ \hline
 &  \\ \hline 
x & x  \\
\hline
\multicolumn{2}{|c|}{$\epsilon = 0$} \\ \hline
\end{tabular} \newline
\begin{tabular}{|c | c| }
 \cline{1-1}  \\ \hline
x &  \\ \hline 
x &   \\
\hline
\multicolumn{2}{|c|}{$\epsilon = 1$} \\ \hline
\end{tabular}
\begin{tabular}{|c | c| }
 \cline{1-1}  \\ \hline
 & x \\ \hline 
 &  x \\
\hline
\multicolumn{2}{|c|}{$\epsilon = 1$} \\ \hline
\end{tabular}
\begin{tabular}{|c | c| }
 \cline{1-1}  \\ \hline
x &  \\ \hline 
 & x   \\
\hline
\multicolumn{2}{|c|}{$\epsilon = 1$} \\ \hline
\end{tabular} \newline
\begin{tabular}{|c | c| }
 \cline{1-1}  \\ \hline
 & x \\ \hline 
x &   \\
\hline
\multicolumn{2}{|c|}{$\epsilon = 1$} \\ \hline
\end{tabular}
\begin{tabular}{|c | c| }
 \cline{1-1} x \\ \hline
 &  \\ \hline 
x &   \\
\hline
\multicolumn{2}{|c|}{$\epsilon = 2$} \\ \hline
\end{tabular}
\begin{tabular}{|c | c| }
 \cline{1-1} x \\ \hline
 &  \\ \hline 
 & x  \\
\hline
\multicolumn{2}{|c|}{$\epsilon = 2$} \\ \hline
\end{tabular} \newline
\begin{tabular}{|c | c| }
 \cline{1-1}  \\ \hline
x & x \\ \hline 
 &   \\
\hline
\multicolumn{2}{|c|}{$\epsilon = 2$} \\ \hline
\end{tabular}
\begin{tabular}{|c | c| }
 \cline{1-1}  \\ \hline
xx &  \\ \hline 
 &   \\
\hline
\multicolumn{2}{|c|}{$\epsilon = 2$} \\ \hline
\end{tabular}
\begin{tabular}{|c | c| }
 \cline{1-1}  \\ \hline
 & xx \\ \hline 
 &   \\
\hline
\multicolumn{2}{|c|}{$\epsilon = 2$} \\ \hline
\end{tabular} \newline
\begin{tabular}{|c | c| }
 \cline{1-1}x  \\ \hline
x &  \\ \hline 
 &   \\
\hline
\multicolumn{2}{|c|}{$\epsilon = 3$} \\ \hline
\end{tabular}
\begin{tabular}{|c | c| }
 \cline{1-1}x  \\ \hline
 & x \\ \hline 
 &   \\
\hline
\multicolumn{2}{|c|}{$\epsilon = 3$} \\ \hline
\end{tabular}
\begin{tabular}{|c | c| }
 \cline{1-1}x x \\ \hline
 &  \\ \hline 
 &   \\
\hline
\multicolumn{2}{|c|}{$\epsilon = 4$} \\ \hline
\end{tabular} \newline
Sehingga fungsi partisinya adalah
\begin{align}
\boldsymbol{Z}& = 3e^0 + 4 e^{-\epsilon /kT} + 5 e^{- 2 \epsilon /kT} + \nonumber \\
& \hspace{0.3 cm } 2e^{- 3 \epsilon /kT} + e^{-4 \epsilon /kT}
\end{align}
Dengan demikian 
\begin{align}
\overline{E}& = \frac{kT^2}{\boldsymbol{Z}} (0+ \frac{4\epsilon}{kT^2} e^{-\epsilon /kT} + \frac{10 \epsilon}{kT^2} e^{- 2 \epsilon /kT} + \nonumber \\
& \hspace{0.3 cm } \frac{6 \epsilon }{kT^2} e^{-3 \epsilon /kT} + \frac{4 \epsilon}{kT^2} e^{- 4 \epsilon}/kT)
\end{align}
Untuk sistem klasik terbedakan, pernyataan $\boldsymbol{ Z} = Z^N $ bisa digunakan, dengan demikain 
\begin{align}
\boldsymbol{Z}& = Z^N \nonumber \\
& = \left(\sum_i g_i e^{-\epsilon_i /kT} \right)^N \nonumber \\
& = \left( g_0 e^{-\epsilon_0 /kT }+ g_1 e^{-\epsilon_1 /kT} + \right.\nonumber 
\\
& \hspace{0.3 cm} \left. g_2 e^{-\epsilon_2 /kT} \right )^2\nonumber \\
& = \left( 2 + 2e^{-\epsilon /kT} + e^{- 2\epsilon /kT} \right)^2  
\end{align}

\end{multicols}
\end{tiny}
\end{document}
