\documentclass[a4paper , 12 pt]{article}
\usepackage[fleqn]{amsmath}
\usepackage{setspace}
%\usepackage[margin =2.5 cm]{geometry }

\onehalfspacing
\pagestyle{empty}
\allowdisplaybreaks

\begin{document}
\centering
JAWABAN SOAL FISIKA  STATISTIK BAB 10 BUKU POINTON 

\begin{enumerate}
\item
	Tunjukkan bahwa  simpangan kuadrat rata-rata dari jumlah 	sistem dalam asembli  memenuhi hubungan berikut
	\begin{enumerate}
	\item untuk statistik Bose-Einstein:
	\[
	\overline{(\delta N)^2} = \sum_i \overline{n_i} (1+ \overline{n_i}) = \sum_i \overline{(\delta n_i )^2}
	\]
	dan 
	\[
	\left \{ \frac{\overline{(\delta n_i)^2}}{\overline{n_i^2}} \right\}= \frac{1}{\overline{n_i}} +1
	\]
	\item Untuk statistik fermi-dirac
	\[
	\overline{(\delta N)^2} = \sum_i \overline{n_i} (1- \overline{n_i}) = \sum_i \overline{(\delta n_i )^2}
	\]
	dan 
	\[
	\left \{ \frac{\overline{(\delta n_i)^2}}{\overline{n_i^2}} \right\}=\frac{1}{\overline{n_i}} - 1
	\]
	\end{enumerate}
	\begin{description}
	\item [Jawab]
	\begin{enumerate}
	\item
		Dari persamaan (10.58) diperoleh
		\begin{align}
		\overline{(\delta N)^2} &  = kT \left \{ \frac{\partial \overline{N}}{\partial \mu } \right \}_{V,T} \nonumber \\
		& = k T \frac{\partial}{\partial \mu} \left \{ \sum_j \overline{n_j} \right  \} \nonumber \\
		& = kT \frac{\partial }{\partial \mu } \left \{ 
		\sum_j \frac{1}{e^{(\epsilon_j -\mu)/kT} - 1}\right  \} \nonumber \\
		& = - kT \sum_j \frac{e^{(\epsilon_j - \mu)/ kT}(-1/kT)}{(e^{(\epsilon_j -\mu)/kT} - 1)^2} \nonumber \\
		& = \sum_j \frac{1}{e^{(\epsilon_j - \mu)/kT} -1} \cdot \frac{e^{(\epsilon_j - \mu)/kT}}{e^{(\epsilon_j - \mu)/kT }-1} \nonumber \\
		& =  \sum_j \frac{1}{e^{(\epsilon_j - \mu)/kT} -1} \cdot  \frac{(e^{(\epsilon_j - \mu)/kT} -1)+1}{e^{(\epsilon_j - \mu)/kT }-1} \nonumber \\
		& =   \sum_j \frac{1}{e^{(\epsilon_j - \mu)/kT} -1} \cdot  \left[ \frac{e^{(\epsilon_j - \mu)/kT }-1}{e^{(\epsilon_j - \mu)/kT }-1} \right. + \nonumber \\  & \hspace{0.3 cm} \left. \frac{1}{e^{(\epsilon_j - \mu)/kT }-1} \right ] 
		\nonumber \\
		& = \sum_j \overline{n_j} (1+ \overline{n_j}) \nonumber \tag{i.1} \\
		& \equiv  \sum_i \overline{(\delta n_j )^2} \nonumber \tag{i.2}
		\end{align}
		Dengan demikian 
		\begin{align}
			\left \{ \frac{\overline{(\delta n_i)^2}}{\overline{n_j^2}} \right\}= \frac{ \sum_j \overline{n_j} (1+ \overline{n_j})}{\overline{n_j^2}} = \frac{1}{\overline{n_j}} +1 \tag{i.3}
		\end{align}
		Persamaan terakhir diperoleh berdasarkan defenisi (i.2) yang diberikan pada soal di buku tersebut, yakni $\overline{n_i}^2 = \overline{n_i^2}$ (QED)
	\item
		Untuk distribusi fermi-dirac cara yang sama dapat dilakukan dengan mengintersek penurunan (i.1) yakni pada langkah ketiga  dengan memasukkan defenisi $n_j$ untuk distribusi fermi dirac yakni
		\[
		 n_j = \frac{1}{e^{(\epsilon_j -\mu)/kT} + 1} \tag{QED}
		\]
	\end{enumerate}
	\end{description}
\item 
	% nomor 2
	Tinjau asembli dari sistem terlokalisir yang satu sama lain saling terbedakan, tunjukkan bahwa sistem ini akan memiliki fungsi grand partisi dalam bentuk
	\begin{align}
	\boldsymbol{Z} & = \sum_{(n_s)} N ! \prod_s \frac{(g_s e^{(\mu - \epsilon_s)/kT})^{ns}}{n_s !} \nonumber \\
	& = (1 - Z e^{\mu/ kT})^{-1} \nonumber
	\end{align}
	Di mana $Z = \displaystyle \sum_s g_s e^{-\epsilon_s /kT }$. Dan   untuk keadaan (asembli) yang diberikan  berlaku $N = \displaystyle \sum_s n_s$
	\newline
	\begin{description}
	\item[Jawab] 
	Untuk sistem partikel klasik terbedakan, maka bobot konfigirasi ke-$i$  dinyatakan oleh persamaan (2.7) yakni:
	\begin{align}
	W_i = N ! \prod_s \left \{ \frac{g_s^{n_s}}{n_s !} \right \}_i \nonumber
	\end{align}
	Sehingga dengan memasukkannya ke persamaan (10.27) akan diperoleh:
	\begin{align}
	\boldsymbol{Z} & = \sum_{(n_s)} N !\left \{ \prod_s \frac{g_s^{n_s}}{n_s !}  \right \} \exp \left[ \sum_s n_s (\mu - \epsilon_s)/kT \right] \nonumber \\
	& = \sum_{(n_s)} N ! \prod_s  \frac{(g_s e^{(\mu - \epsilon_s)/kT})^{n_s}}{n_s !} \tag{ii.1}
	\end{align}
	Sementara dari persamaan (9.21) diperoleh bahwa 
	\begin{align}
	\left(\sum_s x_s \right)^N = \sum_{\left ( \sum_s n_s = N\right )}  N ! \prod_s \frac{x_s^{n_s}}{n_s !} \tag{ii.2}
	\end{align}
	Akan tetapi  karena jumlah partikel dalam ensembel grand kanonik tidak tetap, maka tidak ada batasan $\displaystyle \sum_{(n_s)} = N$ alias nilai $N$ mencakup berapa saja mulai dari $0$ sampai $\infty$ . Maka dengan menggunakan persamaan (ii.2), persamaan (ii.1) dapat dituliskan menjadi 
	\begin{align}
	\boldsymbol{Z} = \sum_0^\infty (Z e^{\mu /kT })^N \tag{ii.3}
	\end{align}
	dengan 
	\[
	Z = \sum_s g_s e^{-\epsilon_s / kT} \tag{ii.4}
	\]
	\newline
	ruas kanan persamaan (ii.3) tidak lain merupakan uraian taylor dari fungsi 
	\[
	(1 - Z e^{\mu /kT})^{-1} \tag{QED}
	\]
	
%	Sementara pada persamaan (10.41) sudah diberikan uraian yang menghasilkan
%	\begin{align}
%	\prod_s (1 - x_s)^{-1} = \sum_{(n_s )} \left(  \prod_s x_s^{n_s} \right ) \tag{ii.1}
%	\end{align}
%	Dalam kasus fungsi grand partisi, $s$ menyatakan konfigurasi asembli ke-$s$ yang masing-masing akan menghasilkan  seperangkat bilangan okupasi yang dinyatakan sebagai kombinasi bilangan $n_s$. Jadi (ii.1) dapat dituliskan menjadi 
%	\begin{align}
%	\boldsymbol{Z} = N ! 
%	\end{align}
%		Untuk sistem partikel tak-terbedakan, maka fungsi grand partisinya sudah diberikan oleh persamaan 10.45. Sehingga untuk sistem partikel terbedakan, maka yang diperlukan hanyalah menambahkan bentuk degenerasi $g_s$ pada pernyataan tersebut yakni:
%		\begin{align}
%		\boldsymbol{Z} =\sum_{(s)} \prod_{s} \left [ 1- g_s e^{ (\mu - \epsilon_s)/kT}\right ]^{-1}
%		\end{align}
%		Karena dalam sistem partikel terbedakan, 
	\end{description}
\item
	% nomor 3
	Dengan menggunakan persamaan 10.12 tunjukkan bahwa potensial kimia dari molekul pada gas sempurna semi klasik adalah
	\begin{align}
	-k T \log \left [ \frac{V}{N} 
	 \left \{ \frac{2 \pi m k T}{h^2} \right \}^{3/2} 
	\right ] \nonumber
	\end{align}
	\begin{description}
	\item[Jawab]
	Dari persamaan (10.12) diketahui bahwa 
	\[
	\alpha = \frac{\mu}{kT} \tag{iii.1}
	\]
	Sementara dari persamaan (7.4) pada kasus gas sempurna semi-klasik, berlaku
	\[
	e^\alpha = A \Rightarrow  \alpha = \log \left(\frac{N}{Z}\right) 
	\]
	Mengingat $A = N/Z$. Dengan demikian
	\[
	-\alpha = \log \left(\frac{Z}{N}\right) = - \frac{\mu}{kT}
	\]
	Sementara dari persaman (7.28) diketahui bahwa
	\[
	Z = \frac{V}{h^3} (2\pi m kT)^{3/2}
	\]
	Maka 
	\begin{align}
	\mu & = - kT \log \left[\frac{V}{N}  \frac{(2\pi m kT)^{3/2}}{h^3} \right]   \nonumber \\
	& = - kT \log \left[  \frac{V}{N} \left\{  \frac{2\pi m kT}{h^2}\right \}^{3/2} \right ] \tag{QED}
	\end{align}
	\end{description}
\item 
	Gunakan persamaan 10.21b dan 10.22 untuk memperoleh pernyataan dari entropi sebagai bentuk fungsi grand partisi. Bandingkan hasilnya dengan yang diperoleh pada persamaan 9.13
	\begin{description}
	\item[Jawab]
	\begin{align}
	S & = \left\lbrace \frac{\partial (pV)}{\partial T} \right \rbrace_{T, \mu} \nonumber  \\
	& = \frac{\partial}{\partial T} (k T \log Z) \nonumber \\
	& = k \left[ \frac{\partial T}{\partial T} \log Z + \frac{\partial}{\partial T} (\log Z) \cdot T  \right ]  \tag{QED}
	\end{align} 
	\end{description}
	
\end{enumerate}


\end{document}}