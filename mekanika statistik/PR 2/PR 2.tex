\documentclass[a4paper , 12 pt]{article}
\usepackage[fleqn]{amsmath}
\usepackage{setspace}
%\usepackage[margin =2.5 cm]{geometry }
%\title{PR 2 MEKANIKA STATISTIK}
%\author{MOHAMMAD FAJAR \newline 20211019}
\onehalfspacing
\pagestyle{empty}
\allowdisplaybreaks
%\pagestyle{empty}

\begin{document}
%\maketitle
\begin{singlespace}

\begin{center}
\large{PR 2 MEKANIKA STATISTIK}

disusun oleh 

\underline{MOHAMMAD FAJAR}

20211019

\end{center}
\end{singlespace}
\begin{enumerate}
% nomor 1
\item Dalam sebuah kotak ada 6 status keadaan yang bisa dimiliki oleh setiap partikel di dalamnya dan antar partikel tidak saling berinteraksi. Asumsikan besar energi tiap status tersebut = 0. 
\begin{enumerate}
\item Jika hanya ada 1 partikel dalam kotak tersebut, tuliskan fungsi partisi kanoniknya. 
\item Jika terisi dua partikel terbedakan, tuliskan fungsi partisi kanoniknya.
\item Jika terisi dua boson identik, tuliskan fungsi partisi kanoniknya. 
\item Jika terisi dua fermion identik, tuliskan fungsi partisi kanoniknya. 
\item Jika terisi dua partikel tak terbedakan (identik), apakah berlaku partisi kanonik sistemnya $Z = \displaystyle Z_1^2 / 2! $ dengan $Z_1$ adalah fungsi partisi kanonik 1 partikel. Jelaskan. 
\end{enumerate}
% jawaban nomor 1 
\textbf{Jawab:}
\begin{enumerate}
\item Untuk satu partikel maka fungsi partisi kanoniknya adalah
\[
Z = 6\, e^{- 0/kT} = 6
\]
\item Untuk dua partikel terbedakan
\[
\boldsymbol{Z} = Z_1 \cdot Z_2 = 36 
\]
\item Untuk 2 boson identik, jika dinotasikan [A,B] sebagai posisi partikel 1 di A dan posisi partikel 2 di B di mana [A,B] = [B,A] dan pada boson satu keadaan bisa ditempati dua partikel maka terdapat 6 kombinasi di mana kedua partikel beradan pada keadaan  yang sama. Sementara untuk kasus tiap partikel berada pada keadaan yang berbeda kombinasinya adalah[1,2], [1,3], [1,4], [1,5], [1,6], [2,3], [2,4], [2,5], [2,6], [3,4], [3,5], [3,6], [4,5], [4,6], [5,6].
sehingga fungsi partisi kanonik adalah:
\[
\boldsymbol{Z} = (6 + 15)e^{0} = 21  
\]
\item Karena pada fermion partikel tidak boleh menempati dua keadaan yang sama, maka  keadaan yang mungkin adalah [1,2], [1,3], [1,4], [1,5], [1,6], [2,3], [2,4], [2,5], [2,6], [3,4], [3,5], [3,6], [4,5], [4,6], [5,6]. sehingga fungsi partisi total adalah:
\[
\boldsymbol{Z} = 15e^{0} = 15 
\]
\item Pernyataan $Z = Z_1^2/ 2!$ dengan $Z_1$ fungsi partisi 1 partikel adalah asumsi teringkas untuk menggambarkan keadaan pada kasus gas ideal dengan densitas yang tidak terlalu besar sehingga peluang seluruh partikel untuk menempati keadaan yang sama sangat kecil sehingga tidak dimasukkan ke dalam perhitungan. Nyatanya adalah rumusan tersebut membagi hasil perhitungan untuk partikel terbedakan dengan memperhatikan adanya fakta bahwa dalam menghitung kombinasi partikel terbedakan dua kedaan yang   berbeda [A,B] dan [B,A] hakikatnya adalah sama sehingga nilainya dibagi dua, padahal pada kasus dua partikel berada pada keadaan yang sama itu keadaannya unik yang mana tidak dicakup dalam rumusan tersebut. 
\end{enumerate}
% nomor 2
\item Dalam sebuah kotak energi yang diperbolehkan untuk dimiliki satu partikel adalah $E_N  = n \alpha$ dengan $n = 0,1,2, ...$ dan $\alpha = \text{konstanta}$. Dalam kotak tersebut terdapat lima partikel yang tidak saling berinteraksi. Ada 3 kasus berbeda yang akan dibahas: (i) boson identik, (ii) fermion identik, (iii) partikel terbedakan. 
\begin{enumerate}
\item Untuk masing-masing ketiga kasus tersebut, hitunglah energi sistem dalam keadaan dasar (ground state)
\item Misal energi total sistem $E = \alpha $, untuk masing-masing ketiga kasus, hitunglah berapa banyak status sistem yang berbeda yang mungkin. 
\item Ulangi (b) untukl $E = 2 \alpha $ dan $E = 3\alpha$ 
\end{enumerate}
\textbf{Jawab:}
\begin{enumerate}
\item Pada keadaan dasar, kelima boson akan menempati tingkat energi $E_0 = 0$ sehingga total energinya adalah 0.

Pada kasus fermion, seluruh partikel akan menempati 5 tingkat energi pertama yakni $E_n = n\alpha$ dengan $n = 0..5$. Jadi energi totalnya adalah $(0 + 1 + 2 + 3 + 4 + 5)\,\alpha = 15 \, \alpha$. 

Untuk kasus partikel terbedakan sama saja dengan pada kasus boson, di mana seluruh partikel boleh berada pada level energi yang sama sehingga total energinya 0. 
\item Ketika total energi $E =  \alpha$, maka kombinasi yang mungkin untuk boson hanya satu keadaan yakni 4 partikel di level $E = 0$ dan 1 partikel berada pada $E_1 = \alpha$. 

Untuk fermion, mengingat tidak boleh satu level energi  ditempati lebih dari satu partikel (mengigat soal tidak berkaitan dengan fakta real yakni adanya spin), maka tidak ada konfigurasi yang berkaitan dengan total energi, $E = \alpha$.

Untuk partikel terbedakan, konfigurasi keadaan yang mungkin untuk total energi $E_n = \alpha$ adalah 5 mengingat masing-masing partikel mendapat giliran untuk menempati level energi $E_n = \alpha$. 
\item Untuk boson konfigurasi yang mungkin untuk total energi $E = 2 \, \alpha$ jika keadaan makrokanonik  dinotasikan sebagai [Jumlah partikel pada level energi 1, Jumlah partikel pada level energi 2, ... dst] adalah 
[3,2] dan [4,0,1]. 
Untuk total energi $E = 3\, \alpha$ maka konfigurasinya adalah [3,1,1], [4,0,0,1], dan [2,3].

Untuk fermion, sama saja dengan jawaban soal (b) yakni tidak ada konfigurasi yang berkaitan dengan total energi $E_n = 2\,\alpha$ dan  $E_n = 3\,\alpha$.

Untuk partikel terbedakan dengan  total energi $E = 2\, \alpha$,  keadaan mikrokanonik untuk keadaan makrokanonik [3,2] adalah [CDE,AB], [DBE,AC], [BCE,AD], [BCD,AE], [ADE,BC], [ACE,BD], [ACD,BE], [ABE,CD], [ABD,CE], [ABC,DE] yang totalnya ada 10. 
Sementara untuk keadaan makrokanonik [4,0,1] totalnya ada 5 mengingat masing-masing partikel bergiliran menempati level energi $E_n = 2$.

Untuk total energi $E = 3\, \alpha$, keadaan mikrokanonik unutuk keadaan makrokanonik [3,1,1] adalah [CDE,B,A], [BDE,C,A], [BCE,D,A], [BCD,E,A], [CDE,A,B], [ADE,C,B], [ACE,D,B], [ACD,E,B], [BDE,A,C], [ADE,B,C], [ABE,D,C], [ABD,E,C], [BCE,A,D], [ACE,B,D], [ABE,C,D], [ABC,E,D], [BCD,A,E], [ACD,B,E], [ABD,C,E], [ABC,D,E]. Jadi total ada 20 keadaan. 

Untuk keadaan makrokanonik [2,3], keadaan mikrokanoniknya adalah [AB,CDE], [AC,BDE], [AD,BCE], [AE,BCD], [BC,AED], [BD,ACE], [BE,ADE], [CD,ABE], [CE,ABD], [DE,ABC] jadi totalnya ada 10. 

Untuk keadaan makrokanonik [4,0,0,1] total keadaan mikrokanoniknya adalah 5 mengingat masing-masing partikel bergantian menempati level energi $E_n = 3 \, \alpha$.
\end{enumerate}
%nomor 3
\item Fermion dalam sebuah kotak memiliki 2 level keadaan A dan B masing-masing memiliki energi (untuk satu fermion):
\[
E_A = \mu + x  \hspace{1 cm} \text{dan} \hspace{1 cm } E_B = \mu + x 
\]
\begin{enumerate}
\item Hitunglah probabilitas bahwa level A terisi sebuah fermion 
\item Hitunglah probabilitas bahwa level B tidak terisi fermion
\end{enumerate}
% jawaban nomor 3
\textbf{Jawab:}
\begin{enumerate}
\item Fungsi partisi untuk satu biji fermion adalah
\[
Z = e^{-(\mu + x)/kT} + e^{-(\mu - x)/kT}
\]
sehingga peluang level A untuk ditempati fermion adalah
\[
\frac{e^{-(\mu + x)/ kT}}{Z} = \frac{e^{-(\mu + x)/ kT}}{e^{-(\mu + x)/kT} + e^{-(\mu - x)/kT}}
\]
\item Karena keadaan sistem hanya ada dua, maka peluang untuk tidak ditempatinya keadaan B sama saja dengan peluang untuk ditempatinya keadaan A yang sudah dijawab pada point a.
\end{enumerate}
% nomor 4
\item Molekul diatomik \newline
Tinjau molekul diatomik dengan massa masing-masing atom $m_1$ dan $m_2$  dan terpisah sejauh $r_0$. Energi rotasinya diberikan oleh:
\[
E_\mathrm{rot} = \frac{\hbar}{2 I_m} J(J+1)
\]
Dengan $J = 0,1,2, ... $ serta $I_m$ adalah momen inersia molekul tersebut terhadap sumbu rotasinya yaitu. 
\[
I_m = \frac{m_1 m_2}{m_1 + m_2} r_0^2
\]
degenerasi setiap status keadaan $J$ adalah $2J + 1$ 
\begin{enumerate}
\item Jika didefinisikan, $\theta_r = \displaystyle \frac{\hbar^2}{2 k I_m}$, tuliskan fungsi partisi kanonik 1 molekul untuk rotasi $Z_\mathrm{rot}$ 
\item  Jika $T$ atau $I_m $ sangat kecil, berilah ungkapan aproksimasi untuk $Z_\mathrm{rot}$.
\item Jika $T$ cukup tinggi dan $I_m$ tidak kecil, turunkan ungkapan bagi $Z_\mathrm{rot}$
\end{enumerate}
% jawaban nomor 4
\textbf{Jawab:}
\begin{enumerate}
\item 
\begin{align}
Z_\mathrm{rot} & = \sum_s  g_s e^{-\epsilon/ kT} \nonumber \\
& = \sum_{j = 0}^{\infty} (2J + 1)\exp  \left[ - \frac{\hbar^2}{2 \frac{m_1 m_2}{m_1 + m_2} r_0^2} J (J+1)/ kT\right] \nonumber \\
& = \sum_{J = 0}^{\infty} (2J + 1) \exp \left(-  \theta_r J(J +1)/T\right) \nonumber 
\end{align}
\item Untuk $T$ atau $I_m$ sangat  kecil maka $1/ T$ atau $1/ I_m$ akan sangat besar dan tentunya nilai exponensialnya akan sangat-sangat besar dengan demikian nilai fungsi partisi rotasinya akan hampir habis atau $Z_\mathrm{rot} \approx 0$.
\item untuk $T$ sangat besar dan $I_m $ tidak cukup kecil, maka kita dapat melakukan ekspansi fourier terhadap nilai exponensialnya yakni:
\begin{align}
Z_\mathrm{rot}& = \sum_{J = 0}^{\infty} (2 J + 1) \frac{1}{e^{\theta_r J(J+1)/ T}}\nonumber \\
&  =  \sum_{J = 0}^{\infty}  (2 J + 1) \frac{1}{1  + \frac{\theta_r J(J+1)}{T} + \frac{1}{2} \left(\frac{\theta_r J(J+1)}{T}\right)^2 + ...} \nonumber \\
& \approx \sum_{J=0}^{\infty} (2 J + 1) \frac{1}{{1  + \frac{\theta_r J(J+1)}{T}}} \nonumber \\
& \approx  \sum_{J=0}^{\infty} (2 J + 1)  \nonumber
\end{align}
%\begin{align}
%Z_\mathrm{rot} \approx 0 \nonumber
%\end{align}
\end{enumerate}
% nomor 5
\item Jika energi vibrasi 1 molekul diberikan oleh $E_n = \hbar \omega (n+ 1/2) $ dengan $n = 0,1,2,...$ 
\begin{enumerate}
\item Tuliskan fungsi partisi kanonik 1 molekul, jika $ \theta_V = \frac{\hbar \omega}{k} $
\item Hitunglah energi rata-ratanya jika terdapat N buah molekul yang saling tidak berinteraksi. 
\item Turunkan ungkapan bagi kapasitas kalor vibrasi pada volume tetapnya, $C_v$. 
\item Pada temperatur tinggi $ \frac{\theta_\nu}{k} \ll 1$ turunkan ungkapan bagi energi rata-rata dari $C_v$ vibrasi. 
\end{enumerate}
% jawaban no. 5
\textbf{Jawab:}
\begin{enumerate}
\item Fungsi partisinya dinyatakan oleh
\begin{align}
Z &= \sum_{n = 0}^{\infty} e^{-(n+ 1/2)\hbar\omega /kT} \nonumber \\
& = e^{- 1/2 \hbar\omega /k T}\sum_{n=0}^{\infty} e^{-n \hbar\omega /kT} \nonumber \\
& = \frac{e^{-1/2 \hbar\omega / kT}}{(1 - e^{-\hbar\omega /kT})} \nonumber  \\
& = \frac{e^{- \theta_\nu /(2T)}}{(1 - e^{-\theta_\nu/ T})} \nonumber
\end{align}
\item energi rata-ratanya dinyatakan oleh
\begin{align}
\overline{\epsilon} & = kT^2 \left( \frac{\partial \ln Z}{\partial T}\right) \nonumber \\
& = \hbar \omega \left(\frac{1}{2} + \frac{1}{e^{\hbar \omega /k T} - 1}\right) \nonumber 
\end{align}
\item Kapasitas kalor pada volume tetap dinyatakan oleh
\begin{align}
C_v &= \frac{\partial U}{\partial T} = N \frac{\partial \overline{\epsilon}}{\partial T} \nonumber \\
& = N \hbar \omega \frac{(\hbar \omega /(kT^2) )e^{\hbar \omega /kT}}{(e^{\hbar \omega / k T } - 1)^2} \nonumber \\
& = N\,k \, \theta_\nu  \frac{(\theta_\nu/T^2) e^{\theta_\nu/ T}}{(e^{\theta_\nu/kT} - 1)^2} \nonumber
\end{align}
\item Untuk tempertur tinggi ($ \frac{\theta_\nu}{T} \ll 1$) maka  $1  \approx 1 + \frac{\theta_\nu}{2T} - \frac{\theta_\nu}{2T} - \left( \frac{\theta_\nu}{2T}\right)^2  = (1+  \frac{\theta_\nu}{2T})(1 -   \frac{\theta_\nu}{2T})$ sehingga nilai energi rata-ratanya adalah
\begin{align}
\overline{\epsilon} & \approx \hbar \omega \left( \frac{1}{2} + \frac{1}{1+\frac{\hbar \omega}{k T} +\frac{1}{2} \left(\frac{\hbar \omega}{k T}\right)^2 - 1}\right) \nonumber \\
 & = \hbar \omega \left( \frac{1}{2} + \frac{1}{\frac{\hbar \omega}{k T} + \frac{1}{2}\left(\frac{\hbar \omega}{k T}\right)^2}  \right) \nonumber \\
 & = \hbar \omega \left( \frac{1}{2} + \frac{1}{\frac{\theta_\nu}{T} + \frac{1}{2}\left(\frac{\theta_\nu}{T}\right)^2}\right)\nonumber \\
 & = \hbar \omega \left( \frac{1}{2} + \frac{1}{\frac{\theta_\nu}{T} \left(1+ \frac{1}{2}\frac{\theta_\nu}{T}\right)}\right) \nonumber \\
 & \approx \hbar \omega \left( \frac{1}{2} + \frac{ (1+  \frac{\theta_\nu}{2T})(1 -   \frac{\theta_\nu}{2T})}{\frac{\theta_\nu}{T} \left(1+ \frac{1}{2}\frac{\theta_\nu}{T}\right)} \right) \nonumber \\
 & = \hbar \omega \left( \frac{1}{2} + \frac{T}{\theta_\nu} \left(1 - \frac{\theta_\nu }{2 T}\right)\right) \nonumber \\
 & = \frac{\hbar \omega T}{\theta_\nu} \nonumber \\
 & = k T \nonumber
\end{align}
Sementara nilai kapasitas kalor pada volume tetap adalah
\begin{align}
 C_v & = N\,k \, \theta_\nu  \frac{(\theta_\nu/T^2) e^{\theta_\nu/ T}}{(e^{\theta_\nu/kT} - 1)^2} \nonumber \\
 & \approx Nk \left( \frac{\theta_\nu}{T}\right)^2 \frac{1 + \frac{\theta_\nu}{T}+ \frac{1}{2}\left(\frac{\theta_\nu}{T}\right)^2}{\left(\frac{\theta_\nu}{T} + \frac{1}{2} \left(\frac{\theta_\nu}{T}\right)^2 \right)^2} \nonumber \\
 & = Nk \frac{1}{\left(\frac{T}{\theta_\nu}\right)}\cdot \frac{1 + \frac{\theta_\nu}{T}+ \frac{1}{2}\left(\frac{\theta_\nu}{T}\right)^2}{\left(\frac{\theta_\nu}{T} + \frac{1}{2} \left(\frac{\theta_\nu}{T}\right)^2 \right)^2} \nonumber  \\
 & = N k \frac{1 + \frac{\theta_\nu}{T}+ \frac{1}{2}\left(\frac{\theta_\nu}{T}\right)^2}{1 + \frac{\theta_\nu}{T}+ \frac{1}{4}\left(\frac{\theta_\nu}{T}\right)^2} \nonumber \\
 &  \approx Nk  \nonumber
\end{align}
\end{enumerate}
\end{enumerate}
\end{document}